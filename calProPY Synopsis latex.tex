\documentclass{article}
\usepackage{graphicx}
\usepackage{float}
\usepackage{listings}
\usepackage{color}
\usepackage{xcolor}

\title{\textbf{Synopsis Calculator with UI using Python}}
\author{Sushant Bhourjar (d10)}


\begin{document}
\maketitle

\section{Introduction}
In this project, we aim to develop a calculator application with a graphical user interface (GUI) using Python and Tkinter. The calculator will be designed to perform basic arithmetic operations such as addition, subtraction, multiplication, and division, making it a user-friendly tool for performing mathematical calculations.

\section{Objective}
The primary objectives of this project are as follows:
\begin{itemize}
    \item To create a user-friendly calculator with a graphical user interface.
    \item To implement basic arithmetic operations (addition, subtraction, multiplication, division).
    \item To provide a clear and intuitive user interface for input and displaying results.
    \item To enhance user experience by incorporating user-friendly design and functionality.
\end{itemize}

\section{Methodology}
The project will be implemented using the following methodology:

\begin{itemize}
    \item \textbf{Python Programming:} We will use the Python programming language for the core logic of the calculator.
    \item \textbf{Tkinter Library:} The Tkinter library will be utilized to create the graphical user interface for the calculator.
    \item \textbf{UI Design:} The calculator's user interface will be designed with user experience in mind, ensuring that it is easy to use and visually appealing.
    \item \textbf{Arithmetic Operations:} We will implement the basic arithmetic operations (addition, subtraction, multiplication, division) in the Python code.
    \item \textbf{Event Handling:} Event handling will be used to capture user input and perform calculations based on the user's selections.
\end{itemize}

\section{Features}
The calculator will include the following features:

\begin{itemize}
    \item Buttons for digits 0-9.
    \item Buttons for addition, subtraction, multiplication, and division.
    \item A clear button to reset the input.
    \item An equals (=) button to calculate the result.
    \item A display area to show the input and the calculated result.
\end{itemize}

\section{Conclusion}
By the completion of this project, we will have a fully functional calculator with a user-friendly graphical user interface. This calculator will be capable of performing basic arithmetic operations, and its design will prioritize an intuitive user experience.
\section{output}
\begin{left}
	\includegraphics[scale=0.4]{calProSS.png}
\end{left}
\end{document}
